\documentclass[journal,12pt,twocolumn]{IEEEtran}
%
\usepackage{setspace}
\usepackage{gensymb}
%\doublespacing
\singlespacing

%\usepackage{graphicx}
%\usepackage{amssymb}
%\usepackage{relsize}
\usepackage[cmex10]{amsmath}
%\usepackage{amsthm}
%\interdisplaylinepenalty=2500
%\savesymbol{iint}
%\usepackage{txfonts}
%\restoresymbol{TXF}{iint}
%\usepackage{wasysym}
\usepackage{amsthm}
\usepackage{cite}
\usepackage{cases}
\usepackage{subfig}
%\usepackage{xtab}
\usepackage{longtable}
\usepackage{multirow}
%\usepackage{algorithm}
%\usepackage{algpseudocode}
\usepackage{booktabs}
\usepackage{enumitem}
\usepackage{mathtools}
\usepackage{tikz}
\usepackage{pgfplots}
\usepackage{circuitikz}
\usepackage{verbatim}
\usepackage{tfrupee}
\usepackage[breaklinks=true]{hyperref}
%\usepackage{stmaryrd}
\usepackage{tkz-euclide} % loads  TikZ and tkz-base
%\usetkzobj{all}
\usetikzlibrary{fit}
\usetikzlibrary{calc,math}
%\pgfdeclarelayer{background}
%\pgfsetlayers{background}
\usepackage{listings}
\usepackage{color}                                            %%
\usepackage{array}                                            %%
\usepackage{longtable}                                        %%
\usepackage{calc}                                             %%
\usepackage{multirow}                                         %%
\usepackage{hhline}                                           %%
\usepackage{ifthen}                                           %%
%optionally (for landscape tables embedded in another document): %%
\usepackage{lscape}     
\usepackage{multicol}
\usepackage{chngcntr}
%\usepackage{enumerate}

%\usepackage{wasysym}
%\newcounter{MYtempeqncnt}
\DeclareMathOperator*{\Res}{Res}
%\renewcommand{\baselinestretch}{2}
\renewcommand\thesection{\arabic{section}}
\renewcommand\thesubsection{\thesection.\arabic{subsection}}
\renewcommand\thesubsubsection{\thesubsection.\arabic{subsubsection}}

\renewcommand\thesectiondis{\arabic{section}}
\renewcommand\thesubsectiondis{\thesectiondis.\arabic{subsection}}
\renewcommand\thesubsubsectiondis{\thesubsectiondis.\arabic{subsubsection}}

% correct bad hyphenation here
\hyphenation{op-tical net-works semi-conduc-tor}
\def\inputGnumericTable{}                                 %%

\lstset{
	%language=C,
	frame=single, 
	breaklines=true,
	columns=fullflexible
}
%\lstset{
	%language=tex,
	%frame=single, 
	%breaklines=true
	%}

\begin{document}
	%
	
	\bibliographystyle{IEEEtran}
	%\bibliographystyle{ieeetr}
	\providecommand{\mbf}{\mathbf}
	\providecommand{\pr}[1]{\ensuremath{\Pr\left(#1\right)}}
	\providecommand{\qfunc}[1]{\ensuremath{Q\left(#1\right)}}
	\providecommand{\sbrak}[1]{\ensuremath{{}\left[#1\right]}}
	\providecommand{\lsbrak}[1]{\ensuremath{{}\left[#1\right.}}
	\providecommand{\rsbrak}[1]{\ensuremath{{}\left.#1\right]}}
	\providecommand{\brak}[1]{\ensuremath{\left(#1\right)}}
	\providecommand{\lbrak}[1]{\ensuremath{\left(#1\right.}}
	\providecommand{\rbrak}[1]{\ensuremath{\left.#1\right)}}
	\providecommand{\cbrak}[1]{\ensuremath{\left\{#1\right\}}}
	\providecommand{\lcbrak}[1]{\ensuremath{\left\{#1\right.}}
	\providecommand{\rcbrak}[1]{\ensuremath{\left.#1\right\}}}
	\providecommand{\dec}[2]{\ensuremath{\overset{#1}{\underset{#2}{\gtrless}}}}
	\newcommand{\myvec}[1]{\ensuremath{\begin{pmatrix}#1\end{pmatrix}}}
	\newcommand{\mydet}[1]{\ensuremath{\begin{vmatrix}#1\end{vmatrix}}}
	\newcommand*{\permcomb}[4][0mu]{{{}^{#3}\mkern#1#2_{#4}}}
	\newcommand*{\perm}[1][-3mu]{\permcomb[#1]{P}}
	\newcommand*{\comb}[1][-1mu]{\permcomb[#1]{C}}
	\title{
		AI1110: Assignment 6
	}
	\author{
		SADINENI ABHINAY - CS21BTECH11055
	}
	
	\maketitle
	\begin{abstract}
		This document contains the solution to Question of Chapter 13 (Probability) in the NCERT Class 12 Textbook.
	\end{abstract}
	
	\textbf{Probability  ex 13.1 q5.}

   if $\pr{A}=\frac{6}{11}$ ,$\pr{B}=\frac{5}{11}$  and $\pr{A+ B}=\frac{7}{11}$ ,find
   \begin{enumerate}
   	\item $\pr{AB}$
   	\item$\pr{A|B}$
   	\item$\pr{B|A}$
   \end{enumerate}
\textbf{Solution.}
	Let X,Y are random variables that represents the occurence of events A and B.
Given $\pr{X=1}=\frac{6}{11}$ ,$\pr{Y=1}=\frac{5}{11}$ and $\pr{\cbrak{X=1} +\cbrak{Y=1}}=\frac{7}{11}$
	\begin{table}[ht!]
		\centering
		\input{tables/Table1.tex}
		\caption{Events}
		\label{table:1}
	\end{table}
\begin{enumerate}
\item Intersection 
\begin{align}
	\pr{AB}&=\pr{\cbrak{X=1},\cbrak{Y=1}}
	\\
     &=\pr{X=1}+\pr{Y=1}
     \\
     & -\pr{\cbrak{X=1}+\cbrak{Y=1}}    
     \\
     &=\frac{6}{11}+\frac{5}{11}-\frac{7}{11}
     \\
     &=\frac{4}{11}     
\end{align}
	\item Conditional probability
	\begin{align}
		\pr{A|B}&=\frac{\pr{\cbrak{X=1},\cbrak{Y=1}}}{\pr{Y=1}}\\
		        &=\frac{\frac{4}{11}}{\frac{5}{11}}\\
		        &=\frac{4}{5}
	\end{align}
\item Conditional probability
	\begin{align}
	\pr{B|A}&=\frac{\pr{\cbrak{X=1},\cbrak{Y=1}}}{\pr{X=1}}\\
	&=\frac{\frac{4}{11}}{\frac{6}{11}}\\
	 &=\frac{2}{3}
   \end{align}
	\end{enumerate}

    
\end{document}