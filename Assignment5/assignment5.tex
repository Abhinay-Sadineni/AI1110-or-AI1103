\documentclass[journal,12pt,twocolumn]{IEEEtran}
%
\usepackage{setspace}
\usepackage{gensymb}
%\doublespacing
\singlespacing

%\usepackage{graphicx}
%\usepackage{amssymb}
%\usepackage{relsize}
\usepackage[cmex10]{amsmath}
%\usepackage{amsthm}
%\interdisplaylinepenalty=2500
%\savesymbol{iint}
%\usepackage{txfonts}
%\restoresymbol{TXF}{iint}
%\usepackage{wasysym}
\usepackage{amsthm}
\usepackage{cite}
\usepackage{cases}
\usepackage{subfig}
%\usepackage{xtab}
\usepackage{longtable}
\usepackage{multirow}
%\usepackage{algorithm}
%\usepackage{algpseudocode}
\usepackage{booktabs}
\usepackage{enumitem}
\usepackage{mathtools}
\usepackage{tikz}
\usepackage{pgfplots}
\usepackage{circuitikz}
\usepackage{verbatim}
\usepackage{tfrupee}
\usepackage[breaklinks=true]{hyperref}
%\usepackage{stmaryrd}
\usepackage{tkz-euclide} % loads  TikZ and tkz-base
%\usetkzobj{all}
\usetikzlibrary{fit}
\usetikzlibrary{calc,math}
%\pgfdeclarelayer{background}
%\pgfsetlayers{background}
\usepackage{listings}
\usepackage{color}                                            %%
\usepackage{array}                                            %%
\usepackage{longtable}                                        %%
\usepackage{calc}                                             %%
\usepackage{multirow}                                         %%
\usepackage{hhline}                                           %%
\usepackage{ifthen}                                           %%
%optionally (for landscape tables embedded in another document): %%
\usepackage{lscape}     
\usepackage{multicol}
\usepackage{chngcntr}
%\usepackage{enumerate}

%\usepackage{wasysym}
%\newcounter{MYtempeqncnt}
\DeclareMathOperator*{\Res}{Res}
%\renewcommand{\baselinestretch}{2}
\renewcommand\thesection{\arabic{section}}
\renewcommand\thesubsection{\thesection.\arabic{subsection}}
\renewcommand\thesubsubsection{\thesubsection.\arabic{subsubsection}}

\renewcommand\thesectiondis{\arabic{section}}
\renewcommand\thesubsectiondis{\thesectiondis.\arabic{subsection}}
\renewcommand\thesubsubsectiondis{\thesubsectiondis.\arabic{subsubsection}}

% correct bad hyphenation here
\hyphenation{op-tical net-works semi-conduc-tor}
\def\inputGnumericTable{}                                 %%

\lstset{
	%language=C,
	frame=single, 
	breaklines=true,
	columns=fullflexible
}
%\lstset{
	%language=tex,
	%frame=single, 
	%breaklines=true
	%}

\begin{document}
	%
	
	\bibliographystyle{IEEEtran}
	%\bibliographystyle{ieeetr}
	\providecommand{\mbf}{\mathbf}
	\providecommand{\pr}[1]{\ensuremath{\Pr\left(#1\right)}}
	\providecommand{\qfunc}[1]{\ensuremath{Q\left(#1\right)}}
	\providecommand{\sbrak}[1]{\ensuremath{{}\left[#1\right]}}
	\providecommand{\lsbrak}[1]{\ensuremath{{}\left[#1\right.}}
	\providecommand{\rsbrak}[1]{\ensuremath{{}\left.#1\right]}}
	\providecommand{\brak}[1]{\ensuremath{\left(#1\right)}}
	\providecommand{\lbrak}[1]{\ensuremath{\left(#1\right.}}
	\providecommand{\rbrak}[1]{\ensuremath{\left.#1\right)}}
	\providecommand{\cbrak}[1]{\ensuremath{\left\{#1\right\}}}
	\providecommand{\lcbrak}[1]{\ensuremath{\left\{#1\right.}}
	\providecommand{\rcbrak}[1]{\ensuremath{\left.#1\right\}}}
	\providecommand{\dec}[2]{\ensuremath{\overset{#1}{\underset{#2}{\gtrless}}}}
	\newcommand{\myvec}[1]{\ensuremath{\begin{pmatrix}#1\end{pmatrix}}}
	\newcommand{\mydet}[1]{\ensuremath{\begin{vmatrix}#1\end{vmatrix}}}
	\newcommand*{\permcomb}[4][0mu]{{{}^{#3}\mkern#1#2_{#4}}}
	\newcommand*{\perm}[1][-3mu]{\permcomb[#1]{P}}
	\newcommand*{\comb}[1][-1mu]{\permcomb[#1]{C}}
		\title{
				AI1110: Assignment 5
		}
		\author{
			SADINENI ABHINAY - CS21BTECH11055
		}
			
	\maketitle
	\begin{abstract}
		This document contains the solution to Question of Chapter 16 (Probability) in the NCERT Class 11 Textbook.
	\end{abstract}
	
	\textbf{Probability 16.3 Example 8.}
A coin is tossed three times, consider the following events.
A: ‘No head appears’, B: ‘Exactly one head appears’ and C: ‘Atleast two heads
appear’.
Do they form a set of mutually exclusive and exhaustive events?
	
	\textbf{Solution.}
	Let $X_i\in \cbrak{0,1}, i = 1, 2, 3$ represent a coin toss, or, the Bernoulli random variable.  Then the outcome of the game is
	If 
	\begin{align}
		\pr{X_i = 1} &= p,
		\\
		\pr{X = k} &= \comb{n}{k}p^k\brak{1-p}^{n-k}, \quad k = 0,\dots, n
	\end{align}
	$X$ is known as a Binomial random variable.  For the given problem, $n = 3, p = \frac{1}{2}$
		\begin{align}
		X = X_1+X_2+X_3	
	\end{align}
Now let us calculate the probabilities of each events
	\begin{table}[ht!]
		\centering
		{\columnwidth}{!} {
	\begin{tabular}{|c|c|c|c|c|c|}
		\hline
		x &1 &2&3 & 4& 5 \\
		\hline
		y &7 & 6 & 5 &4 & 3\\
		\hline
	\end{tabular}
}

		\caption{Events}
		\label{Table:1}
	\end{table}
	\begin{align}
		\pr{A}&=\pr{X=0}
		\\
	     	  &=\comb{3}{0}\brak{\frac{1}{2}}^3 
	     	  \\
	     	  &=\frac{1}{8}
	\end{align}
	\begin{align}     	  	
	    \pr{B}&=\pr{X=1}\\
	          &=\comb{3}{1}\brak{\frac{1}{2}}^3 \\
	          &=\frac{3}{8}
	\end{align}
	\begin{align}	          		        
	    \pr{C}&= \pr{X \ge 2} \\
	    &= \comb{3}{2}\brak{\frac{1}{2}}^3+\comb{3}{3}\brak{\frac{1}{2}}^3
	    \\
	    &= \frac{1}{2}
	\end{align}
    The sum of probabilities
    \begin{align}
	    \pr{A}+\pr{B}+\pr{C}&=\frac{1}{8}+\frac{3}{8}+ \frac{1}{2}\\
	                        &=1\label{eq:eq14}
	\end{align}
The probabilities where both events happen at same time
\begin{align}
	\pr{AB}&=\pr{\cbrak{X=0}\cbrak{X=1}}\\
	       &=0\label{eq:eq16}\\
	 \pr{BC}&=\pr{\cbrak{X=1}\cbrak{X\ge2}}\\
	 &=0\label{eq:eq18}\\
	 \pr{AC}&=\pr{\cbrak{X=0}\cbrak{X\ge2}}\\
	 &=0\label{eq:eq20}
\end{align}
	by results \eqref{eq:eq14},\eqref{eq:eq16},\eqref{eq:eq18},\eqref{eq:eq20}
	we can prove that the events are mutually exculsive and exhaustive.
\end{document}