\documentclass{beamer}
\usepackage{listings}
\usepackage{blkarray}
\usepackage{listings}
\usepackage{subcaption}
\usepackage{url}
\usepackage{tikz}
\usepackage{tkz-euclide} % loads  TikZ and tkz-base
%\usetkzobj{all}
\usetikzlibrary{calc,math}
\usepackage{float}
\newcommand\norm[1]{\left\lVert#1\right\rVert}
\renewcommand{\vec}[1]{\mathbf{#1}}
\usepackage[export]{adjustbox}
\usepackage[utf8]{inputenc}
\usepackage{amsmath}
\usepackage{amsfonts}
\usepackage{tikz}
\usepackage{hyperref}
\usepackage{bm}
\hypersetup{
	colorlinks = true,
	linkbordercolor = {white},
	linkcolor={red},
	citecolor={green},
	filecolor={blue},
	menucolor={red},
	runcolor={cyan},
	urlcolor={blue},
	breaklinks=true
}
\usetikzlibrary{automata, positioning}
\usetheme{Boadilla}
\providecommand{\pr}[1]{\ensuremath{\Pr\left(#1\right)}}
\providecommand{\mbf}{\mathbf}
\providecommand{\qfunc}[1]{\ensuremath{Q\left(#1\right)}}
\providecommand{\sbrak}[1]{\ensuremath{{}\left[#1\right]}}
\providecommand{\lsbrak}[1]{\ensuremath{{}\left[#1\right.}}
\providecommand{\rsbrak}[1]{\ensuremath{{}\left.#1\right]}}
\providecommand{\brak}[1]{\ensuremath{\left(#1\right)}}
\providecommand{\lbrak}[1]{\ensuremath{\left(#1\right.}}
\providecommand{\rbrak}[1]{\ensuremath{\left.#1\right)}}
\providecommand{\cbrak}[1]{\ensuremath{\left\{#1\right\}}}
\providecommand{\lcbrak}[1]{\ensuremath{\left\{#1\right.}}
\providecommand{\rcbrak}[1]{\ensuremath{\left.#1\right\}}}
\providecommand{\abs}[1]{\vert#1\vert}

\newcounter{saveenumi}
\newcommand{\seti}{\setcounter{saveenumi}{\value{enumi}}}
\newcommand{\conti}{\setcounter{enumi}{\value{saveenumi}}}

\makeatletter
\newenvironment<>{proofs}[1][\proofname]{%
	\par
	\def\insertproofname{#1\@addpunct{.}}%
	\usebeamertemplate{proof begin}#2}
{\usebeamertemplate{proof end}}
\makeatother

\title{AI1110 Assignment 6}
\author{SADINENI ABHINAY}
\date{CS21BTECH11055}
\begin{document}
	
	\begin{frame}
		\titlepage
	\end{frame}

	\begin{frame}{Abstract}
		\begin{itemize}
			\item 	This document contains the solution to Question of Chapter 13 (Probability) in the NCERT Class 12 Textbook.
		\end{itemize}
	\end{frame}
	
	\begin{frame}{Question}
		\begin{block}{\textbf{Probability  ex 13.1 q5.}}
			 if $\pr{A}=\frac{6}{11}$ ,$\pr{B}=\frac{5}{11}$  and $\pr{A+ B}=\frac{7}{11}$ ,find
			\begin{enumerate}
				\item $\pr{AB}$
				\item$\pr{A|B}$
				\item$\pr{B|A}$
			\end{enumerate}
		\end{block}
	\end{frame}
	

	
	\begin{frame}{Theory}
			 \begin{block}{Inclusive and exculsive principle}
			If there are events A and B and the individuial probabilities ,probability of occurrence of both events at same time are known,then probability of occurence of either event A or B is 
			\begin{align}
				\pr{A+B}=\pr{A}+\pr{B}-\pr{AB}
				\end{align}
			\end{block}
			\begin{block}{Conditional probablity}
			‘the probability of A given B'
			\begin{align}
				\pr{A|B}=\frac{\pr{AB}}{\pr{B}}
			\end{align}
		\end{block}
			   
	\end{frame}
	\begin{frame}{Solution}
		Let X,Y are random variables that represents the occurence of events A and B. In the problem it is given that 
			$\pr{X=1}=\frac{6}{11},\pr{Y=1}=\frac{5}{11}$ and $\pr{\cbrak{X=1} +\cbrak{Y=1}}=\frac{7}{11}$
	\begin{table}[ht!]
	\centering
	 {
	\begin{tabular}{|c|c|}
		\hline
		\textbf{Probability}&\textbf{Value}\\
		\hline
		$\pr{X=0}$ &0.016 \\
		\hline
	$\pr{X=1}$ &0.094 \\
		\hline
	$\pr{X=2}$ &0.234 \\
		\hline
	$\pr{X=3}$ &0.312 \\
		\hline
	$\pr{X=4}$ &0.234 \\
	       \hline
	$\pr{X=5}$ &0.094 \\
	       \hline
	$\pr{X=6}$ &0.016 \\
	       \hline
	\end{tabular}
}

	\caption{Events}
	\label{table:1}
    \end{table}
	\end{frame}
	\begin{frame}{Solution}
		\begin{enumerate}
			\item Intersection of A and B
			\begin{align}
				\pr{AB}&=\pr{\cbrak{X=1},\cbrak{Y=1}}
				\\
				&=\pr{X=1}+\pr{Y=1}-\pr{\cbrak{X=1}+\cbrak{Y=1}}    
				\\
				&=\frac{6}{11}+\frac{5}{11}-\frac{7}{11}
				\\
				&=\frac{4}{11}     
			\end{align}
			\item Conditional probability of A given B
			\begin{align}
				\pr{A|B}&=\frac{\pr{\cbrak{X=1},\cbrak{Y=1}}}{\pr{Y=1}}\\
				&=\frac{\frac{4}{11}}{\frac{5}{11}}\\
				&=\frac{4}{5}
			\end{align}
		\seti
		\end{enumerate}
	\end{frame}
\begin{frame}{Solution}
	\begin{enumerate}
		\conti
		\item Conditional probability  of B given A
	\begin{align}
		\pr{B|A}&=\frac{\pr{\cbrak{X=1},\cbrak{Y=1}}}{\pr{X=1}}\\
		&=\frac{\frac{4}{11}}{\frac{6}{11}}\\
		&=\frac{2}{3}
	\end{align}
\end{enumerate}
\end{frame}
\end{document}
