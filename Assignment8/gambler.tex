\documentclass{beamer}
\usetheme{CambridgeUS}
\usepackage{listings}
\usepackage{blkarray}
\usepackage{listings}
\usepackage{subcaption}
\usepackage{url}
\usepackage{tikz}
\usepackage{tkz-euclide} % loads  TikZ and tkz-base
%\usetkzobj{all}
\usetikzlibrary{calc,math}
\usepackage{float}
\renewcommand{\vec}[1]{\mathbf{#1}}
\usepackage[export]{adjustbox}
\usepackage[utf8]{inputenc}
\usepackage{amsmath}
\usepackage{amsfonts}
\usepackage{tikz}
\usepackage{hyperref}
\usepackage{bm}
\usetikzlibrary{automata, positioning}
\providecommand{\pr}[1]{\ensuremath{\Pr\left(#1\right)}}
\providecommand{\e}[1]{\ensuremath{E\left(#1\right)}}
\providecommand{\mbf}{\mathbf}
\providecommand{\qfunc}[1]{\ensuremath{Q\left(#1\right)}}
\providecommand{\sbrak}[1]{\ensuremath{{}\left[#1\right]}}
\providecommand{\lsbrak}[1]{\ensuremath{{}\left[#1\right.}}
\providecommand{\rsbrak}[1]{\ensuremath{{}\left.#1\right]}}
\providecommand{\brak}[1]{\ensuremath{\left(#1\right)}}
\providecommand{\lbrak}[1]{\ensuremath{\left(#1\right.}}
\providecommand{\rbrak}[1]{\ensuremath{\left.#1\right)}}
\providecommand{\cbrak}[1]{\ensuremath{\left\{#1\right\}}}
\providecommand{\lcbrak}[1]{\ensuremath{\left\{#1\right.}}
\providecommand{\rcbrak}[1]{\ensuremath{\left.#1\right\}}}
\providecommand{\abs}[1]{\vert#1\vert}
\newcommand*{\permcomb}[4][0mu]{{{}^{#3}\mkern#1#2_{#4}}}
\newcommand*{\perm}[1][-3mu]{\permcomb[#1]{P}}
\newcommand*{\comb}[1][-1mu]{\permcomb[#1]{C}}
\renewcommand{\thetable}{\arabic{table}} 
\newcounter{saveenumi}
\newcommand{\seti}{\setcounter{saveenumi}{\value{enumi}}}
\newcommand{\conti}{\setcounter{enumi}{\value{saveenumi}}}
\usepackage{amsmath}
\setbeamertemplate{caption}[numbered]{}
\DeclareUnicodeCharacter{2212}{\textendash}
\title{AI1110 Assignment 8}
\author{SADINENI ABHINAY-CS21BTECH11055 }
\date{\today}
\logo{\large \LaTeX{}}
\begin{document}
	
	\begin{frame}
		\titlepage
	\end{frame}

\begin{frame}{Outline}
  \tableofcontents
\end{frame}
\section{Abstract}
\begin{frame}{Abstract}
This document contains gambler's ruin problem of chapter 3 in papoullis textbook
\end{frame}

\section{Problem}
\begin{frame}{Problem}
Two players A and B plays game consecutively till one of them loses all his capital.
Suppose A starts with a capital of \$a and B with a capital of \$b and the loser pays \$1
to the winner in each game. Let p represent the probability of winning each game for A
and $q = 1 - p$ for player B. Find the probability of ruin for each player if no limit is set
for the number of games
Let $N_a$ denote the average duration of the
game for player A starting with capital a. Show that 
\begin{equation}
N_a =
\begin{cases}
\frac{b}{2p-1}-\frac{a+b}{2p-1}\frac{1-\brak{\frac{p}{q}}^{b}}{1-\brak{\frac{p}{q}}^{a+b}} & ,p \neq q \\
ab  &,p=q=\frac{1}{2}
\end{cases}
\end{equation}
\end{frame}

\section{Solution}
\begin{frame}{Solution}
Let $P_{n}$ denote the probability of the event $X_{n}$ is "A's ultimate ruin when his wealth is \$n $\brak{0 \le n \le a+b}$". A will be ruined in two ways , increase his wealth by \$1 the lose ruin later or reduce his wealth by \$1 ruin later,This can be recursive written as:
Let H be event:"A succeds in the next game"
Let D represent the duration of the game.
\begin{align}
X_n  &= X_n.\brak{H+\bar{H}}=X_n H +X_n\bar{H} \\
P_n  &=\pr{X_n}=\pr{X_n|H}\pr{H}+\pr{X_n|\bar{H}}\pr{\bar{H}}\\
 &=pP_{n+1}+qP_{n-1}
\\
  P_0&= 1 ,P_{a+b}=0
\end{align}

\end{frame}
\section{Solution}
\begin{frame}{}
Let us solve the recursion:
\begin{align}
 \brak{P_{n+1}-P_n}&=\frac{q}{p}\brak{P_{n}-P_{n-1}} \\
                   &=\brak{\frac{q}{p}}^{n}\brak{P_1-1} 
\end{align}
try to use the initial conditions:
\begin{align}
P_{a+b}-P_{n}=\sum_{k=n}^{a+b-1} P_{k+1}-P_{k}&=\sum_{k=n}^{a+b-1}\brak{\frac{q}{p}}^{k}\brak{P_1-1} \\
&=\brak{P_{1} -1}\frac{\brak{\frac{q}{p}}^{n}-\brak{\frac{q}{p}}^{a+b}}{1-\frac{q}{p}}
\end{align}
\end{frame}
\begin{frame}{}
Since $P_{a+b}=0$ and $P_{0}=1$
\begin{align}
P_{n}=\brak{1-P_{1}}\frac{\brak{\frac{q}{p}}^{n}-\brak{\frac{q}{p}}^{a+b}}{1-\frac{q}{p}} \\
P_{0}=1=\brak{1-P_{1}}\frac{\brak{\frac{q}{p}}^{0}-\brak{\frac{q}{p}}^{a+b}}{1-\frac{q}{p}}
\end{align}
Divide the last equations:
\begin{align}
P_{n}=\frac{\brak{\frac{q}{p}}^{n}-\brak{\frac{q}{p}}^{a+b}}{1-\brak{\frac{q}{p}}^{a+b}}
\end{align}
if $p=q=\frac{1}{2}$ ,then $P_n,P_{n-1}...$ are in A.P and equal
\begin{align}
P_n=\frac{b}{a+b}
\end{align}
\end{frame}
\begin{frame}{}
Now let us calculate the duration of the game:
As we know its a recursion, with initial conditions $N_{0}=N_{a+b}=0$,
\begin{align}
N_k&=\pr{H}\e{D|H}+\pr{\bar{H}}\e{D|\bar{H}}\\
   &=p\brak{1+N_{k+1}}+q\brak{1+N_{k-1}}\\
   &=1+pN_{k+1}+qN_{k-1}
\end{align}
This is inhomogeneous linear difference equation. General solution:
\begin{align}
N_k=Ck+A+B\brak{\frac{q}{p}}^{k}
\end{align}
\end{frame}
\begin{frame}
On using $N_{0}=N_{a+b}=0$:
\begin{align}
C&=\frac{1}{1-2p}\\
B&=\frac{a+b}{\brak{1-2p}\brak{1-\brak{\frac{q}{p}}^{a+b}}}\\
A&=\frac{-B}{\brak{1-\brak{\frac{q}{p}}^{a+b}}}
\end{align}
On solving equation:
\begin{equation}
N_a =
\begin{cases}
\frac{b}{2p-1}-\frac{a+b}{2p-1}\frac{1-\brak{\frac{p}{q}}^{b}}{1-\brak{\frac{p}{q}}^{a+b}} & ,p \neq q \\
ab  &,p=q=\frac{1}{2}
\end{cases}
\end{equation}
\end{frame}
\end{document}
