\documentclass[journal,12pt,twocolumn]{IEEEtran}

\usepackage{enumitem}
\usepackage{amsmath}
\usepackage{amssymb}
\usepackage{gensymb}
\usepackage{graphicx}

\def\inputGnumericTable{}

\usepackage[latin1]{inputenc}                                 
\usepackage{color}                                            
\usepackage{array}                                            
\usepackage{longtable}                                        
\usepackage{calc}                                             
\usepackage{multirow}                                         
\usepackage{hhline}                                           
\usepackage{ifthen}
\usepackage{minibox}
\usepackage{caption} 
\usepackage{listings}
\captionsetup[table]{skip=3pt}  
\providecommand{\pr}[1]{\ensuremath{\Pr\left(#1\right)}}
\renewcommand{\thefigure}{\arabic{table}}
\renewcommand{\thetable}{\arabic{table}}
\lstset{
	%language=C,
	frame=single, 
	breaklines=true,
	columns=fullflexible
}                               


\vspace{3cm}
\title{
	AI1110: Assignment 3
}
\author{ 
	SADINENI ABHINAY - CS21BTECH11055
}	
\begin{document}
	\maketitle
	\begin{abstract}
		This document contains the solution to Question of Chapter 15 (Probability) in the NCERT Class 9 Textbook.
	\end{abstract}
	
	\textbf{Probability Excercise 15.1 Q1.}
	
	In a cricket match, a batswoman hits a boundary 6 times out of 30 balls she plays. Find the probability that she did not hit a boundary.
	
	\textbf{Solution.}
	
	Let $X \in \{0,1\} $is random variable that denote whether the batswomen hits a boundary or not.
	
	\begin{table}[ht!]
		\centering
		{\columnwidth}{!} {
	\begin{tabular}{|c|c|c|c|c|c|}
		\hline
		x &1 &2&3 & 4& 5 \\
		\hline
		y &7 & 6 & 5 &4 & 3\\
		\hline
	\end{tabular}
}

	\end{table}
The probability that the batswomen hits a  boundary:
\begin{align}
	 &\pr{X=1}\\
&=\frac{\text{The number of times when she hit boundary}}{\text{total number of balls played}}\\
	&=\frac{6}{30}\\
	&=0.2
\end{align}
The probability that the batswomen doesn't hit a boundary:
We know that the event mention are mutual exclusive and exhaustive events 
\begin{align}
	\pr{X=0}&=1-\pr{X=1}\\
	&=1-0.2\\
	&=0.8
\end{align}
	$\therefore$ the probability that the batswomen does'nt hit
	 a boundary is 0.8
\end{document}