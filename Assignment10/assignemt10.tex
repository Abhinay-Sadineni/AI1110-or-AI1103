\documentclass{beamer}
\usetheme{CambridgeUS}
\usepackage{listings}
\usepackage{blkarray}
\usepackage{listings}
\usepackage{subcaption}
\usepackage{url}
\usepackage{tikz}
\usepackage{tkz-euclide} % loads  TikZ and tkz-base
%\usetkzobj{all}
\usetikzlibrary{calc,math}
\usepackage{float}
\renewcommand{\vec}[1]{\mathbf{#1}}
\usepackage[export]{adjustbox}
\usepackage[utf8]{inputenc}
\usepackage{amsmath}
\usepackage{amsfonts}
\usepackage{tikz}
\usepackage{hyperref}
\usepackage{bm}
\usetikzlibrary{automata, positioning}
\providecommand{\pr}[1]{\ensuremath{\Pr\left(#1\right)}}
\providecommand{\e}[1]{\ensuremath{E\left(#1\right)}}
\providecommand{\mbf}{\mathbf}
\let\vec\mathbf
\providecommand{\dec}[2]{\ensuremath{\overset{#1}{\underset{#2}{\gtrless}}}}
\newcommand{\myvec}[1]{\ensuremath{\begin{pmatrix}#1\end{pmatrix}}}
\newcommand{\mydet}[1]{\ensuremath{\begin{vmatrix}#1\end{vmatrix}}}
\providecommand{\qfunc}[1]{\ensuremath{Q\left(#1\right)}}
\providecommand{\sbrak}[1]{\ensuremath{{}\left[#1\right]}}
\providecommand{\lsbrak}[1]{\ensuremath{{}\left[#1\right.}}
\providecommand{\rsbrak}[1]{\ensuremath{{}\left.#1\right]}}
\providecommand{\brak}[1]{\ensuremath{\left(#1\right)}}
\providecommand{\lbrak}[1]{\ensuremath{\left(#1\right.}}
\providecommand{\rbrak}[1]{\ensuremath{\left.#1\right)}}
\providecommand{\cbrak}[1]{\ensuremath{\left\{#1\right\}}}
\providecommand{\lcbrak}[1]{\ensuremath{\left\{#1\right.}}
\providecommand{\rcbrak}[1]{\ensuremath{\left.#1\right\}}}
\providecommand{\abs}[1]{\vert#1\vert}
\newcommand*{\permcomb}[4][0mu]{{{}^{#3}\mkern#1#2_{#4}}}
\newcommand*{\perm}[1][-3mu]{\permcomb[#1]{P}}
\newcommand*{\comb}[1][-1mu]{\permcomb[#1]{C}}
\renewcommand{\thetable}{\arabic{table}} 
\newcounter{saveenumi}
\newcommand{\seti}{\setcounter{saveenumi}{\value{enumi}}}
\newcommand{\conti}{\setcounter{enumi}{\value{saveenumi}}}
\usepackage{amsmath}
\setbeamertemplate{caption}[numbered]{}
\DeclareUnicodeCharacter{2212}{\textendash}
\title{AI1110 Assignment 10}
\author{SADINENI ABHINAY-CS21BTECH11055 }
\date{\today}
\logo{\large \LaTeX{}}
\begin{document}
\newcommand{\BEQA}{\begin{eqnarray}}
\newcommand{\EEQA}{\end{eqnarray}}
\newcommand{\define}{\stackrel{\triangle}{=}}
\newcommand*\circled[1]{\tikz[baseline=(char.base)]{
    \node[shape=circle,draw,inner sep=2pt] (char) {#1};}}
\bibliographystyle{IEEEtran}
	\begin{frame}
		\titlepage
	\end{frame}
\begin{frame}{Outline}
\tableofcontents
\end{frame}

\section{Abstract}
\begin{frame}{Abstract}
This document contains problem of chapter 7 in papoullis textbook
\end{frame}
\section{Problem}
\begin{frame}{Problem}
\begin{block}{Chapter 7-7.9}
Show that if 
	\begin{align}
		x_i \ge 0 , \e{x_i^2}=M \text{ and } s=\sum_{i=1}^{n} x_i
	\end{align}
	then,
	\begin{align}
		\e{s^2} \le M\e{n^2}
	\end{align}
\end{block}
\end{frame}
\section{Solution}
\begin{frame}{Solution}
	\begin{enumerate}
	\item Simplfy $\e{s^2}$
\begin{align}
	\e{s^2|n=n}&=\e{\brak{\sum_{i=1}^{n} x_i^2}} \\
	     &=\e{\sum_{j=1}^n\sum_{i=1}^n x_i x_j} \\
	       \end{align}
		\item Triangle Inequality 
			\begin{align}
				\e{x_i x_j}^2 &\le \e{x_i^2}\e{x_j^2}  \\
				             &\le  M^2 
					\end{align}
					\seti
	\end{enumerate}
\end{frame}
\section{Final Result}
\begin{frame}{final Result}

\begin{enumerate}
\seti
\item Let us Apply the Above inequlity in the question
	\begin{align}
\e{\sum_{j=1}^{n} \sum_{i=1}^{n} x_i x_j} &\le M^2 \times n^{2} \\
					\e{s^2}&=\e{\e{s^2|n=n}}\\ &=\e{M^2\brak{n^2}} \\
				       				       &=M^2\e{n^2}
		\end{align}
		\end{enumerate}
\end{frame}
\end{document}
