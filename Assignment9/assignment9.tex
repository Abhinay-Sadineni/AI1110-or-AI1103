\documentclass{beamer}
\usetheme{CambridgeUS}
\usepackage{listings}
\usepackage{blkarray}
\usepackage{listings}
\usepackage{subcaption}
\usepackage{url}
\usepackage{tikz}
\usepackage{tkz-euclide} % loads  TikZ and tkz-base
%\usetkzobj{all}
\usetikzlibrary{calc,math}
\usepackage{float}
\renewcommand{\vec}[1]{\mathbf{#1}}
\usepackage[export]{adjustbox}
\usepackage[utf8]{inputenc}
\usepackage{amsmath}
\usepackage{amsfonts}
\usepackage{tikz}
\usepackage{hyperref}
\usepackage{bm}
\usetikzlibrary{automata, positioning}
\providecommand{\pr}[1]{\ensuremath{\Pr\left(#1\right)}}
\providecommand{\e}[1]{\ensuremath{E\left(#1\right)}}
\providecommand{\mbf}{\mathbf}
\let\vec\mathbf
\providecommand{\dec}[2]{\ensuremath{\overset{#1}{\underset{#2}{\gtrless}}}}
\newcommand{\myvec}[1]{\ensuremath{\begin{pmatrix}#1\end{pmatrix}}}
\newcommand{\mydet}[1]{\ensuremath{\begin{vmatrix}#1\end{vmatrix}}}
\providecommand{\qfunc}[1]{\ensuremath{Q\left(#1\right)}}
\providecommand{\sbrak}[1]{\ensuremath{{}\left[#1\right]}}
\providecommand{\lsbrak}[1]{\ensuremath{{}\left[#1\right.}}
\providecommand{\rsbrak}[1]{\ensuremath{{}\left.#1\right]}}
\providecommand{\brak}[1]{\ensuremath{\left(#1\right)}}
\providecommand{\lbrak}[1]{\ensuremath{\left(#1\right.}}
\providecommand{\rbrak}[1]{\ensuremath{\left.#1\right)}}
\providecommand{\cbrak}[1]{\ensuremath{\left\{#1\right\}}}
\providecommand{\lcbrak}[1]{\ensuremath{\left\{#1\right.}}
\providecommand{\rcbrak}[1]{\ensuremath{\left.#1\right\}}}
\providecommand{\abs}[1]{\vert#1\vert}
\newcommand*{\permcomb}[4][0mu]{{{}^{#3}\mkern#1#2_{#4}}}
\newcommand*{\perm}[1][-3mu]{\permcomb[#1]{P}}
\newcommand*{\comb}[1][-1mu]{\permcomb[#1]{C}}
\renewcommand{\thetable}{\arabic{table}} 
\newcounter{saveenumi}
\newcommand{\seti}{\setcounter{saveenumi}{\value{enumi}}}
\newcommand{\conti}{\setcounter{enumi}{\value{saveenumi}}}
\usepackage{amsmath}
\setbeamertemplate{caption}[numbered]{}
\DeclareUnicodeCharacter{2212}{\textendash}
\title{AI1110 Assignment 9}
\author{SADINENI ABHINAY-CS21BTECH11055 }
\date{\today}
\logo{\large \LaTeX{}}
\begin{document}
\newcommand{\BEQA}{\begin{eqnarray}}
\newcommand{\EEQA}{\end{eqnarray}}
\newcommand{\define}{\stackrel{\triangle}{=}}
\newcommand*\circled[1]{\tikz[baseline=(char.base)]{
    \node[shape=circle,draw,inner sep=2pt] (char) {#1};}}
\bibliographystyle{IEEEtran}
	\begin{frame}
		\titlepage
	\end{frame}


\section{Abstract}
\begin{frame}{Abstract}
This document contains problem of chapter 6 in papoullis textbook
\end{frame}
\section{Problem}
\begin{frame}{Problem}
\begin{block}{Chapter 6-6.56}
x and yare zero mean independent random variables with variances $\sigma_1^2$ and $\sigma_2^2$ respectively,that is , x $\sim N\brak{0,\sigma_1^2 } ,y\sim N\brak{0,\sigma_2^2}$ Let 
\begin{align}
z=ax+by+c \hspace{1cm} c \neq 0 
\end{align}
\begin{enumerate}
\item Find the characterstic equation $\phi_z \brak{u}$ of z.
\item Using $\phi_z \brak{u}$ conclude that z is also a normal random variable.
\item Find the mean and variance of z.
\end{enumerate}
\end{block}
\end{frame}
\section{Theory}
\begin{frame}{Theory}
\begin{block}{Gaussian Normal Distribution }
 A normal distribution (also known as Gaussian, Gauss, or Laplace–Gauss distribution) is a type of continuous probability distribution for a real-valued random variable. The general form of its probability density function is:
 \begin{align}
F(x) = \frac{1}{{\sigma \sqrt {2\pi } }}e^{{{ - \left( {x - \mu } \right)^2 } \mathord{\left/ {\vphantom {{ - \left( {x - \mu } \right)^2 } {2\sigma ^2 }}} \right. \kern-\nulldelimiterspace} {2\sigma ^2 }}}
 \end{align}
\end{block}
\begin{block}{Charteristic Function}
For a scalar random variable X the characteristic function is defined as the expected value of $e^{itX}$,where i is the imaginary unit, and t $\in$ R is the argument of the characteristic function
\end{block}
\end{frame}
\begin{frame}{Theory}
\begin{eqnarray*}
  \mathcal{F}(\omega) &=&\int_{-\infty}^\infty \frac{1}{\sqrt{2\pi}} \mathrm{e}^{-\frac{x^2}{2}} \mathrm{e}^{j \omega x} \mathrm{d} x \\ &=& \int_{0}^\infty \frac{1}{\sqrt{2\pi}} \mathrm{e}^{-\frac{x^2}{2}} \mathrm{e}^{j \omega x} \mathrm{d} x + \int_{-\infty}^0 \frac{1}{\sqrt{2\pi}} \mathrm{e}^{-\frac{x^2}{2}} \mathrm{e}^{j \omega x} \mathrm{d} x \\
   &=& \int_{0}^\infty \frac{1}{\sqrt{2\pi}} \mathrm{e}^{-\frac{x^2}{2}} \mathrm{e}^{j \omega x} \mathrm{d} x + \int_{0}^{\infty} \frac{1}{\sqrt{2\pi}} \mathrm{e}^{-\frac{x^2}{2}} \mathrm{e}^{-j \omega x} \mathrm{d} x \\
   &=&  2 \int_{0}^\infty \frac{1}{\sqrt{2\pi}} \mathrm{e}^{-\frac{x^2}{2}} \cos(\omega x) \mathrm{d} x
  \end{eqnarray*}
  \begin{eqnarray*}
   \mathcal{F}^\prime(\omega) &=& -\frac{2}{\sqrt{2\pi}} \int_0^\infty \mathrm{e}^{-\frac{x^2}{2}} x \sin(\omega x) \mathrm{d} x = \frac{2}{\sqrt{2\pi}} \int_0^\infty \sin(\omega x) \mathrm{d} \left( \mathrm{e}^{-\frac{x^2}{2}} \right) \\
   &=& \frac{2}{\sqrt{2\pi}} \left. \mathrm{e}^{-\frac{x^2}{2}} \sin(\omega x) \right|_0^\infty - \frac{2}{\sqrt{2\pi}} \int_0^\infty \mathrm{e}^{-\frac{x^2}{2}} \omega \cos(\omega x) \mathrm{d} x \\ 
    &=& - \omega \mathcal{F}(\omega) 
  \end{eqnarray*}
\end{frame}
\begin{frame}{}
\begin{block}{Final Result}
The solution to so obtained ODE, $\mathcal{F}'\brak{\omega}=-\mathcal{F}\brak{\omega}$ is\\$\mathcal{F}\brak{\omega}=ce^{-\frac{\omega^2}{2}}$
for normal distribution $\mathcal{N}\brak{\mu,\sigma^2}$ 
\begin{align}
\phi_x\brak{t}=e^{it\mu-\frac{1}{2}\sigma^2t^2}
\end{align}
\end{block}
\end{frame}
\section{Solution}
\begin{frame}{Solution}
\begin{enumerate}
\item The characterstic equation is given by:
\begin{align}
\phi_Z \brak{u} &=\e{e^{juZ}} \\
                &=\e{e^{ju(aX+bY+c)}} \\
                &=\e{e^{ju(aX)}}\e{e^{ju(bY)}}e^{juc} \\
                &=\phi_X \brak{au}\phi_Y \brak{bu}e^{juc} \\
                &=e^{juc-\frac{\brak{(a^2\sigma_1^2+b^2\sigma_2^2)u^2}}{2}}
\end{align}
\seti
\end{enumerate}
\end{frame}{}
\begin{enumerate}
\conti
\item On comparing with  general characterstic equation for normal density We can say: 
\begin{align}
Z \sim \mathcal{N}\brak{c,a^2\sigma_1^2+b^2\sigma_2^2}
\end{align}
\item for normal distribution $\mathcal{N}\brak{\mu,\sigma^2}$ 
\begin{align}
\e{X}=\mu,Variance=\sigma^2
\end{align}
Here for distribution of Z:
\begin{align}
\e{X}=c,Variance=a^2\sigma_1^2+b^2\sigma_2^2
\end{align}
\end{enumerate}
\end{document}
